%!TEX TS-program = xelatex
%!TEX encoding = UTF-8 Unicode

\documentclass[10pt,a4paper]{article}
\usepackage[margin=2.5cm]{geometry}
\usepackage[parfill]{parskip}
\usepackage[superscript,biblabel]{cite}
\usepackage[UKenglish]{babel}
\usepackage[UKenglish]{isodate}
\usepackage{graphicx}

\usepackage{fontspec,xltxtra,xunicode}
\defaultfontfeatures{Mapping=tex-text}
\setmainfont[Mapping=tex-text]{Helvetica Neue}
\setmonofont[Scale=MatchLowercase]{Source Code Pro}
\usepackage{microtype}

\title{Do cGMP-specific phosphodiesterase type 5 inhibitors have a role in the prophylaxis or treatment of High Altitude Pulmonary Oedema?}
\author{Kenrick Turner}
\date{\today}

\begin{document}

\maketitle

% Total: 2500 words
% Close loops: claim, justify, support, implications

\section*{Introduction}

% 375 words

High Altitude Pulmonary Oedema (HAPE) is a non-cardiogenic pulmonary oedema associated with altitude, characterised by the triad of excessive pulmonary artery hypertension (PAH), high protein permeability in pulmonary capillaries, and normal left heart function.\cite{Fred:1962hy,Roy:1969tt,Schoene:1988tz} Originally described by Houston in 1960, it remains a leading cause of preventable death at altitude, with an estimated mortality of 44\% if untreated, and 4--11\% if treated with descent and oxygen.\cite{HOUSTON:1960gz,Lobenhoffer:1982eg,MENON:1965gk} Whilst incidence is generally low -- 0.01\%, 0.2\% and 2\% at high, very high, and extreme altitudes respectively -- it increases dramatically with faster ascents.\cite{Bartsch:2002cg} As tourism to high altitudes approaches 40 million annually, the incidence of HAPE is likely to increase worldwide.\cite{West:ug}

It is now accepted that the ``stress failure'' model of pulmonary oedema described by West \emph{et al.} is most likely to represent the pathogenesis of HAPE in-vivo.\cite{Maggiorini:2001vq} This model hypothesises that an uneven Hypoxic Pulmonary Vasoconstrictor Response (HPVR) occurs due to global hypoxia at altitude, causing uneven perfusion to the lung parenchyma. This results in markedly elevated pulmonary capillary pressures, causing mechanical failure of the capillary wall, allowing protein-rich fluid to leak from the pulmonary capillaries into the alveoli.\cite{West:1991vc,West:1995tg}

As excessive pulmonary artery hypertension appears to be a key step in the pathogenesis of HAPE,\cite{Maggiorini:2001vq,Swenson:2002ub} efforts have been focussed on modulating the HPVR for both prophylaxis and treatment of HAPE. Consequently, the calcium channel antagonist Nifedipine is widely used for this purpose, as an adjunct to descent and oxygen.\cite{Luks:2010ht}

\begin{figure}[!hb]
\centering
\includegraphics[scale=.6]{no-cgmp.pdf}
\caption{The NO-cGMP pathway in the vascular endothelium\cite{Archer:2009cx}}
\label{fig:no-cgmp_pathway}
\end{figure}

cGMP-specific phosphodiesterase type 5 (PDE5) presents an attractive therapeutic target, as PDE5 is abundantly expressed in the vascular endothelium of the pulmonary circulation.\cite{Ahn:1991vt} PDE5 catalyses the hydrolysis of cGMP to GMP, terminating the intracellular secondary messaging cascade from the potent vasodilator nitric oxide (Figure~\ref{fig:no-cgmp_pathway}).\cite{Archer:2009cx} Work in animal models suggests that by inhibiting PDE5, the effects of endogenous nitric oxide are potentiated, resulting in pulmonary vasodilation and a reduction in pulmonary artery pressure, thereby mitigating the key step in the pathogenesis of HAPE.\cite{Zhao:2001kj}

\section*{Methodology}

A targeted search of the literature was conducted during June 2013 using the MEDLINE database. MEDLINE was initially searched using Medical Subject Headings (MeSH) keywords. However the closest term for HAPE was the supplemental concept ``Pulmonary edema of mountaineers [Supplementary Concept]'' and frequently was not associated with relevent papers. Hence, MeSH keywords were abandoned, and an alternative strategy was employed by searching directly for the terms ``High Altitude Pulmonary Edema''[\emph{sic}] and ``Phosphodiesterase 5 Inhibitors''. To further broaden the search scope, the term ``High Altitude Pulmonary Hypertension'' was used synonymously for HAPE, as it is a prerequisite step its pathogenesis.

Additional primary research papers were identified through sources cited in recent meta-analyses and literature reviews in significant journals in the field.

Consensus guidelines, primary research articles and meta-analyses were included, whilst review articles, case reports and case series were excluded. Results were constrained to full-text articles and published in the last decade, but this latter clause was relaxed owing to the relative paucity of published research in this area.

\section*{Results}

A total of seventeen primary research papers, one meta-analysis, and two recent consensus statements were identified using the search strategy described. Four studies were discarded as they were irrelevant to HAPE or PAH.

\section*{Discussion}

\subsection*{Current consensus statements}

\subsection*{HAPE susceptible individuals}

% Brief overview of consensus statements

\subsection*{Pulmonary Artery Hypertension as a surrogate endpoint for HAPE}

Whilst systolic pulmonary artery pressure (sPAP) is traditionally measured with a Swan-Ganz catheter, echocardiographic measurement of the pressure gradient across the tricuspid valve during systole provides an accurate surrogate measurement.\cite{Tramarin:1991uo,Allemann:2000tc}

\subsection*{An effect exaggerated by exercise?}

\subsection*{Choice of drug and dose}

\subsection*{Potential pitfalls}
%Tachyphylaxis
%Responders vs non-responders
%Headache and AMS

\subsection*{Lack of evidence for treatment of HAPE}

\section*{Conclusion}

% 250 words

%headache = common SE of PDE5 inhibitors, also biggest sx of AMS/HACE

\bibliographystyle{naturemag}
\bibliography{hape-essay}

\end{document}