%!TEX TS-program = xelatex
%!TEX encoding = UTF-8 Unicode

\documentclass[10pt,a4paper]{article}
\usepackage[margin=2.5cm]{geometry}
\usepackage[parfill]{parskip}
\usepackage[superscript,biblabel]{cite}
\usepackage[UKenglish]{babel}
\usepackage[UKenglish]{isodate}
\usepackage{graphicx}

\usepackage{fontspec,xltxtra,xunicode}
\defaultfontfeatures{Mapping=tex-text}
\setmainfont[Mapping=tex-text]{Helvetica Neue}
\setmonofont[Scale=MatchLowercase]{Source Code Pro}
\usepackage{microtype}

\title{Do cGMP-specific phosphodiesterase type 5 inhibitors have a role in the prophylaxis or treatment of High Altitude Pulmonary Oedema?}
\author{Kenrick Turner}
\date{\today}

\begin{document}

\maketitle

% Total: 2500 words
% Close loops: claim, justify, support, implications

\section*{Introduction}

% 375 words

High Altitude Pulmonary Oedema (HAPE) is a non-cardiogenic pulmonary oedema associated with altitude, characterised by the triad of excessive pulmonary artery hypertension (PAH), high protein permeability in pulmonary capillaries, and normal left heart function.\cite{Fred:1962hy,Roy:1969tt,Schoene:1988tz} Originally described by Houston in 1960, it remains a leading cause of preventable death at altitude, with an estimated mortality of 44\% if untreated, and 4--11\% if treated with descent and oxygen.\cite{HOUSTON:1960gz,Lobenhoffer:1982eg,MENON:1965gk} Whilst incidence is generally low -- 0.01\%, 0.2\% and 2\% at high, very high, and extreme altitudes respectively -- it increases dramatically with faster ascents.\cite{Bartsch:2002cg} With the rise in tourism and the associated rapid ascent profiles, it would suggest that the incidence of HAPE is likely to increase worldwide.

It is now accepted that the ``stress failure'' model of pulmonary oedema described by West \emph{et al.} is most likely to represent the pathogenesis of HAPE in-vivo.\cite{Maggiorini:2001vq} This model hypothesises that an uneven Hypoxic Pulmonary Vasoconstrictor Response (HPVR) occurs due to global hypoxia at altitude, causing uneven perfusion to the lung parenchyma. This results in markedly elevated pulmonary capillary pressures, causing mechanical failure of the capillary wall, allowing protein-rich fluid to leak from the pulmonary capilleries into the alveoli.\cite{West:1991vc,West:1995tg}

As excessive pulmonary artery hypertension appears to be a key step in the pathogenesis of HAPE, efforts have been focussed on modulating the HPVR for both prophylaxis and treatment of HAPE. Consequently, the calcium channel antagonist Nifedipine is widely used for this purpose, as an adjunct to descent and oxygen.\cite{Luks:2010ht}

\begin{figure}[!hb]
\centering
\includegraphics[scale=.6]{no-cgmp.pdf}
\caption{The NO-cGMP pathway in the vascular endothelium\cite{Archer:2009cx}}
\label{fig:no-cgmp_pathway}
\end{figure}

cGMP-specific phosphodiesterase type 5 (PDE5) presents an attractive therapeutic target, as PDE5 is predominantly expressed in the vascular endothelium of the pulmonary and \emph{corpus cavernosum} circulation. PDE5 catalyses the hydrolysis of cGMP to GMP, terminating the intracellular secondary messanging cascade from the potent vasodilator nitric oxide (Figure~\ref{fig:no-cgmp_pathway}).\cite{Archer:2009cx} It is hypothesised that by inhibiting PDE5, the effects of endogenous nitric oxide are potentiated, resulting in pulmonary vasodilation and a reduction in pulmonary artery pressure.

\section*{Methodology}

A targeted search of the literature was conducted during June 2013 using MEDLINE and Embase databases. MEDLINE was initially searched using Medical Subject Headings (MeSH) keywords. However the closest term for HAPE was the supplemental concept ``Pulmonary edema of mountaineers [Supplementary Concept]'' and frequently was not associated with relevent papers. Hence, MeSH keywords were abandoned, and an alternative strategy was employed by searching directly for the terms ``High Altitude Pulmonary Edema''[\emph{sic}] and ``Phosphodiesterase 5 Inhibitors''. To further broaden the search scope, the term ``High Altitude Pulmonary Hypertension'' was considered to be synonymous with HAPE for the purposes of an initial search. Additional primary research papers were identified through recent meta-analyses and literature reviews in significant journals in the field.

Consensus guidelines, primary research articles and meta-analyses were included, whilst review articles, case reports and case series were excluded. Results were initially constrained to articles published in the last decade, but this clause was removed owing to the relative paucity of published research in this area.

\section*{Results}

\section*{Discussion}

\section*{Recommendations and Conclusion}

% 250 words

\bibliographystyle{naturemag}
\bibliography{hape-essay}

\end{document}