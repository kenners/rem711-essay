%!TEX TS-program = xelatex
%!TEX encoding = UTF-8 Unicode

\documentclass[10pt,a4paper]{article}
\usepackage[margin=2.5cm]{geometry}
\usepackage[parfill]{parskip}
\usepackage[superscript,biblabel]{cite}
\usepackage[UKenglish]{babel}
\usepackage[UKenglish]{isodate}

\usepackage{fontspec,xltxtra,xunicode}
\defaultfontfeatures{Mapping=tex-text}
\setmainfont[Mapping=tex-text]{Helvetica Neue}
\setmonofont[Scale=MatchLowercase]{Inconsolata}
\usepackage{microtype}

\title{Do Phosphodiesterase-5 inhibitors have a role in the prophylaxis or treatment of High Altitude Pulmonary Oedema?}
\author{Kenrick Turner}
\date{\today}

\begin{document}

\maketitle

% Total: 2500 words
% Close loops: claim, justify, support, implications

\section*{Introduction}

% 375 words

High Altitude Pulmonary Oedema (HAPE) is a non-cardiogenic pulmonary oedema associated with altitude, characterised by the triad of excessive pulmonary artery hypertension (PAH), high protein permeability in pulmonary capillaries, and normal left heart function.\cite{Fred:1962hy,Roy:1969tt,Schoene:1988tz} Originally described by Houston in 1960, it remains a leading cause of preventable death at altitude, with an estimated mortality of 44\% if untreated, and 4--11\% if treated with descent and oxygen.\cite{HOUSTON:1960gz,Lobenhoffer:1982eg,MENON:1965gk} Whilst incidence is generally low -- 0.01\%, 0.2\% and 2\% at high, very high, and extreme altitudes respectively -- it increases dramatically with faster ascents.\cite{Bartsch:2002cg} With the rise in tourism and the associated rapid ascent profiles, it would suggest that the incidence of HAPE is likely to increase worldwide.

It is now accepted that the ``stress failure'' model of pulmonary oedema described by West \emph{et al.} is most likely to represent the pathogenesis of HAPE in-vivo.\cite{West:1991vc,Maggiorini:2001vq}

\section*{Methodology}

\section*{Results}

\section*{Discussion}

\section*{Recommendations and Conclusion}

% 250 words

\bibliographystyle{naturemag}
\bibliography{hape-essay}

\end{document}