%!TEX TS-program = xelatex
%!TEX encoding = UTF-8 Unicode

\documentclass[10pt,a4paper]{article}
\usepackage[margin=2.5cm]{geometry}
\usepackage[parfill]{parskip}
\usepackage[superscript,biblabel]{cite}
\usepackage[UKenglish]{babel}
\usepackage[UKenglish]{isodate}
\usepackage{graphicx}

\usepackage[pdfauthor={Kenrick Turner}, pdftitle={Do cGMP-specific phosphodiesterase type 5 inhibitors have a role in the prophylaxis or treatment of High Altitude Pulmonary Oedema?}, pdfsubject={High Altitude Pulmonary Oedema},pdfkeywords={HAPE, PDE5, inhibitors}]{hyperref}

\usepackage{fontspec,xltxtra,xunicode}
\defaultfontfeatures{Mapping=tex-text}
\setmainfont[Mapping=tex-text]{Helvetica Neue}
\setmonofont[Scale=MatchLowercase]{Source Code Pro}
\usepackage{microtype}

\title{Do cGMP-specific phosphodiesterase type 5 inhibitors have a role in the prophylaxis or treatment of High Altitude Pulmonary Oedema?}
\author{Kenrick Turner}
\date{\today}

\begin{document}

\maketitle

% Total: 2500 words
% Close loops: claim, justify, support, implications

\section*{Introduction}

High Altitude Pulmonary Oedema (HAPE) is a non-cardiogenic pulmonary oedema associated with altitude, characterised by the triad of excessive pulmonary artery hypertension (PAH), high protein permeability in pulmonary capillaries, and normal left heart function.\cite{Fred:1962hy,Roy:1969tt,Schoene:1988tz} Originally described by Houston in 1960, it remains a leading cause of preventable death at altitude, with an estimated mortality of 44\% if untreated, and 4--11\% if treated with descent and oxygen.\cite{HOUSTON:1960gz,Lobenhoffer:1982eg,MENON:1965gk} Whilst incidence is generally low -- 0.01\%, 0.2\% and 2\% at high, very high, and extreme altitudes respectively -- it increases dramatically with faster ascents.\cite{Bartsch:2002cg} As tourism to high altitudes approaches 40 million annually, the incidence of HAPE is likely to increase worldwide.\cite{West:ug}

It is now accepted that the ``stress failure'' model of pulmonary oedema described by West \emph{et al.} is most likely to represent the pathogenesis of HAPE in-vivo.\cite{Maggiorini:2001vq} This model hypothesises that an uneven Hypoxic Pulmonary Vasoconstrictor Response (HPVR) occurs due to global hypoxia at altitude, causing uneven perfusion to the lung parenchyma. This results in markedly elevated pulmonary capillary pressures, causing mechanical failure of the capillary wall, allowing protein-rich fluid to leak from the pulmonary capillaries into the alveoli.\cite{West:1991vc,West:1995tg}

As excessive pulmonary artery hypertension appears to be a key step in the pathogenesis of HAPE,\cite{Maggiorini:2001vq,Swenson:2002ub} efforts have been focussed on modulating the HPVR for both prophylaxis and treatment of HAPE. Consequently, the calcium channel antagonist Nifedipine is widely used for this purpose, as an adjunct to descent and oxygen.\cite{Luks:2010ht}

\begin{figure}[!hb]
\centering
\includegraphics[scale=.6]{no-cgmp.pdf}
\caption{The NO-cGMP pathway in the vascular endothelium\cite{Archer:2009cx}}
\label{fig:no-cgmp_pathway}
\end{figure}

cGMP-specific phosphodiesterase type 5 (PDE5) presents an attractive therapeutic target, as PDE5 is abundantly expressed in the vascular endothelium of the pulmonary circulation.\cite{Ahn:1991vt} PDE5 catalyses the hydrolysis of cGMP to GMP, terminating the intracellular secondary messaging cascade from the potent vasodilator nitric oxide (Figure~\ref{fig:no-cgmp_pathway}).\cite{Archer:2009cx} Work in animal models suggests that by inhibiting PDE5, the effects of endogenous nitric oxide are potentiated, resulting in pulmonary vasodilation and a reduction in pulmonary artery pressure, thereby mitigating the key step in the pathogenesis of HAPE.\cite{Zhao:2001kj}

PDE5 inhibitors, such as sildenafil and tadalafil -- well known for their role in erectile dysfunction -- have been in clinical use for over a decade. More recently, they have begun to be used in the treatment of chronic pulmonary hypertension.\cite{NationalPulmonaryHypertensionCentresoftheUKandIreland:2008jh}

\section*{Methodology}

A targeted search of the literature was conducted during June 2013 using the MEDLINE database. MEDLINE was initially searched using Medical Subject Headings (MeSH) keywords. However the closest term for HAPE was the supplemental concept ``Pulmonary edema of mountaineers [Supplementary Concept]'' and frequently was not associated with relevent papers. Hence, MeSH keywords were abandoned, and an alternative strategy was employed by searching directly for the terms ``High Altitude Pulmonary Edema''[\emph{sic}] and ``Phosphodiesterase 5 Inhibitors''. To further broaden the search scope, the term ``High Altitude Pulmonary Hypertension'' was used synonymously for HAPE.

Additional primary research papers were identified through sources cited in recent meta-analyses and literature reviews in significant journals in the field.

Consensus guidelines, primary research articles and meta-analyses were included, whilst review articles, case reports and case series were excluded. Results were constrained to full-text articles and published in the last decade, but this latter clause was relaxed owing to the relative paucity of published research in this area.

\section*{Results}

A total of sixteen randomised double-blinded placebo-controlled trials, one open-label trial, one meta-analysis, and two recent consensus statements were identified using the search strategy described. Four studies were discarded as they were irrelevant to HAPE or PAH.

\section*{Discussion}

\subsection*{A review of the consensus statements}

The current 2010 Wilderness Medical Society consensus guidelines for acute altitude illness contain little regarding PDE5 inhibitors.\cite{Luks:2010ht} The sole recommendation is to suggest tadalafil for prevention of HAPE, though they base this recommendation on a single, small study (n=16).\cite{Maggiorini:2006kz} Surprisingly, the role of sildenafil is not discussed, despite the greater number of studies examining it in the literature.

Intriguingly, the 2008 UK national consensus statement on pulmonary hypertension does not include any guidance for prevention or treatment of either altitude-induced PAH or HAPE.\cite{NationalPulmonaryHypertensionCentresoftheUKandIreland:2008jh} However, when discussing treatment options for Idiopathic and Familial Pulmonary Arterial Hypertension (IPAH/FPAH), the authors discuss the role of PDE5 inhibitors, concluding that they may be used as second-line therapy. This recommendation was derived largely from the results of the SUPER-1 study -- a large (n=278), multi-centre, randomised, placebo-controlled trial comparing sildenafil against placebo -- that demonstrated a reduction in pulmonary artery pressure and increased exercise tolerance.\cite{Galie:2005gx}

Whilst this study's subjects had IPAH and FPAH (\emph{cf.} high-altitude PAH), and was examining the effects of sildenafil over a 12 week period rather than acutely, it supports the earlier animal work by Zhao \emph{et al.}\cite{Zhao:2001kj} that PDE5 inhibitors do indeed reduce pulmonary artery pressures humans, implying that they may have a role in preventing HAPE.

\subsection*{Effects under normobaric hypoxia}

Relatively few studies (3/17) measured clinical HAPE as a primary endpoint -- perhaps owing to its relatively low incidence (2\%) in healthy adults\cite{Bartsch:2002cg}. Instead, most studies use elevated systolic Pulmonary Artery Pressure (sPAP) as a surrogate endpoint for HAPE, given that it is a prerequisite step in its pathogenesis.\cite{Maggiorini:2001vq}

Further, measuring sPAP is non-trivial, as the ``gold-standard'' measurement remains Swan-Ganz catheterisation of the right heart, which is impractical in the setting of high altitude. Thus, some studies have sought to simulate altitude in a laboratory setting through normobaric hypoxia, to permit invasive measurement of sPAP.

Zhao \emph{et al.} used Swan-Ganz catheterisation to observe that a mean rise in sPAP of +9mmHg was almost abolished (+2mmHg) when healthy subjects were exposed to acute normobaric hypoxia (FiO\textsubscript{2}=0.11) when given 100mg Sidenafil 60 minutes beforehand.\cite{Zhao:2001kj}

Using a similar methodology, Kjaergaard \emph{et al.} reported a comparable attenuation in the rise of sPAP in response to normobaric hypoxia (Placebo: +10.9mmHg; Sildenafil: +3.6mmHg; FiO\textsubscript{2}=0.125).\cite{Kjaergaard:2007hp} However, they used echocardiography to measure the pressure gradient across the tricuspid valve during systole as an estimation of sPAP, which in addition to serving as an accurate surrogate, is also practical in non-permissive environments.\cite{Tramarin:1991uo,Allemann:2000tc}

Whilst both these studies reported encouraging results, both were small RCTs (n=10 and n=14 respectively), and further, as demonstrated in animal models, normobaric hypoxia is an inadequate simulation for altitude (hypobaric hypoxia) as pulmonary fluid balance heavily depends on barometric pressure.\cite{Bland:1977kz,Levine:1988uq}

\subsection*{Effects under hypobaric hypoxia}


\subsection*{Prophylaxis for HAPE susceptible individuals}

As described in the literature, the three determinants for HAPE are altitude, speed of ascent, and individual susceptibility.\cite{Bartsch:2001kc,Bartsch:2002cg,Dehnert:2005ca} Of these, individual factors appear to be the strongest predictor of HAPE,\cite{Bartsch:2002cg} introducing the concept that there is a population of ``HAPE-susceptible'' individuals.\cite{Bartsch:2001kc,Dehnert:2005ca}. Evidence from recent studies suggests that for HAPE susceptible individuals, PDE5 inhibitors to confer a benefit.

In 2006, Maggiorini \emph{et al.} conducted a double-blinded, placebo-controlled RCT comparing tadalafil against dexamethasone, and placebo, in preventing HAPE in susceptible individuals.\cite{Maggiorini:2006kz} The authors recruited subjects who had at least one previous episode of HAPE, and after commencing prophylactic medication 24 hours before ascent, took subjects to 4,559m for two days, ascending over 24 hours. Notably, this was the only RCT in the literature that used HAPE (diagnosed by either the Lake Louise clinical criteria\cite{Sutton:1991wx} or chest radiograph) as a primary endpoint.

Whilst in the placebo group, seven of nine subjects (78\%) developed HAPE, only one of eight subjects (13\%) in the tadalafil group was likewise affected. The rise in sPAP (measured with echocardiography) was also significantly blunted (p<0.05), supporting this trend: the mean rise in sPAP in the placebo group was 28mmHg, compared to 13mmHg in the tadalafil group.

A similar study conducted by Fischler \emph{et al.} in 2009 investigating exercise capacity in HAPE-susceptible subjects corroborated this finding. Using an identical regime of tadalafil, ascent rate, and altitude, HAPE was observed in seven of eight placebo subjects (88\%) compared to just one of seven tadalafil subjects (14\%). Likewise, Fischler \emph{et al.} reported a comparable statistically significant reduction in the rise in sPAP between tadalafil and placebo groups.

Both of these studies demonstrate a statistically significant reduction in the incidence of HAPE in susceptible individuals with a regime of 10mg tadalafil b.d. commenced prior to ascent. Additionally, these studies also strengthen the correlation between an excessive rise in sPAP and the manifestation of HAPE.

However, both trials had small study arms, and Maggiorini \emph{et al.} note that they were unable to recruit their target of 18 subjects, as required by their power calculation. Furthermore, two subjects in their tadalafil group were withdrawn as they developed Acute Mountain Sickness (AMS), but the authors undertook an intention-to-treat analysis which still demonstrated a significant difference between the groups. Therefore, in HAPE susceptible individuals PDE5 inhibitors do appear to confer a prophylactic benefit.

\subsection*{An effect exaggerated by exercise?}

\subsection*{Choice of drug and dose}

\subsection*{Potential pitfalls}
%Tachyphylaxis
%Responders vs non-responders
%Headache and AMS

\subsection*{Strength in numbers}

A well-written meta-analysis was conducted by Jin \emph{et al.} in 2010, reviewing randomised controlled trials (RCTs) published before June 2009 on PDE5 inhibitors for high-altitude pulmonary artery hypertension.\cite{Jin:2010fc} They pooled data from ten studies, concluding that PDE5 inhibitors reduced sPAP by a weighted mean of 7.51mmHg (95\% CI 4.16--10.87mmHg; p<0.0001; n=79). Notably, the authors employed Cochrane Collaboration meta-analysis review methodology and only included studies whose methodology used hypobaric hypoxia (\emph{cf.} normobaric hypoxia), with all study altitudes >2,500m.

However, one study included in the meta-analysis had recruited participants exclusively from highland areas (2500--4000m) and was designed to look at the long-term effect of sildenafil, measuring results after a period of 3 months.\cite{Aldashev:2005fr} Another of the trials had only recruited ``HAPE-susceptible'' individuals i.e. patients who had previously had an episode of HAPE.\cite{Maggiorini:2006kz} Further, the wide variation in the studies' choice of drug (sildenafil or tadalafil), dosing regime, altitude, and crucially, speed of ascent to altitude, limit the utility of this meta-analysis' conclusions.

\subsection*{Lack of evidence for treatment of HAPE}

\section*{Conclusion}

% 250 words

%headache = common SE of PDE5 inhibitors, also biggest sx of AMS/HACE

\bibliographystyle{naturemag}
\bibliography{hape-essay}

\end{document}